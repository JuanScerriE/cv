\documentclass[12pt]{article}

% PACKAGES

\usepackage[
top = 1.5cm,
bottom = 1.5cm,
left = 1.65cm,
right = 1.6cm,
marginparsep = 0pt,
marginparwidth = 0pt,
]{geometry} % for good margins
\usepackage{ulem, contour} % for proper underline
\usepackage{titlesec} % modify sections
\usepackage{titling} % better titles
\usepackage{xcolor} % colored text
\usepackage[utf8]{inputenc} % poor attempt at unicode support
\usepackage{setspace} % control spacing
\usepackage{parskip} % no indentation
\usepackage{nopageno} % no page numbers
\usepackage{hyperref} % for hyperlinks

% MACROS & DEFS

\def\CC{{C\nolinebreak[4]\hspace{-.05em}\raisebox{.45ex}{\tiny\bf ++}}}
\newcommand{\dateentry}[4]{
\begin{tabular}{r||l}
\begin{minipage}[t]{0.2\textwidth}\hfill#1\end{minipage} & \begin{minipage}[t]{0.70\textwidth}\textbf{#2}

\textit{#3}

\small{#4}\hfill\end{minipage}
\end{tabular}
}

\definecolor{OneBlue}{RGB}{69, 154, 224}
\definecolor{OneBlack}{RGB}{40, 44, 52}
\definecolor{OneGrey}{RGB}{100, 100, 100}

\DeclareRobustCommand{\ul}[1]{%
	\uline{\phantom{#1}}%
	\llap{\contour{white}{#1}}%
}

\renewcommand{\ULdepth}{1.8pt}
\contourlength{0.8pt}

\newcommand{\ulsubsection}[1]{\subsection{\ul{#1}}}
\newcommand{\ulbf}[1]{\textbf{\ul{#1}}}

\renewcommand{\familydefault}{\sfdefault}

\titleformat{\section}
{\color{OneBlue}\bfseries\Large}
{}
{0em}
{}[\color{black}\titlerule]

\titlespacing{\subsection}
{0cm}
{.5cm}
{.25cm}

\titleformat{\subsection}[runin]
{\bfseries}
{}
{0cm}
{}[\hspace{0.25cm}---]

\author{Juan Scerri}

\renewcommand{\maketitle}{
\begin{minipage}{3in}
\begin{spacing}{1.5}
\begin{flushleft}
{\Huge\bfseries\theauthor} \\
{\large\itshape\uppercase{Curriculum Vitae}}
\end{flushleft}
\end{spacing}
\end{minipage}
\hfill
\begin{minipage}{3.75in}
\begin{spacing}{1.25}
\begin{footnotesize}
\color{OneGrey}
\begin{flushright}
\ul{Address}: St. Anthony, Flat 1, Morphou Street, \.Zurrieq, Malta \\
\ul{Phone}: +356 9911 5740 \\
\ul{Email}: juan.scerri.21@um.edu.mt
\end{flushright}
\end{footnotesize}
\end{spacing}
\end{minipage}
\vspace{-0.25in}
}

\begin{document}

%----------------------------------------------------------------------------------------
%	RESUME
%----------------------------------------------------------------------------------------

\maketitle % print the title

\section{Relevant Experience}

\dateentry{Q4 2021}{HackTheBox University CTF}{Team:
MelitaBusters, Ranking: 59/594}{I participated as a member of the
University of Malta's CTF team, with a focus on cryptography.}

\dateentry{Q3 2021}{MITA Student Placement Programme}{Deployment
at the IT Management Unit of the Ministry for National Heritage,
Arts and Local Government}{I was tasked with checking local
  councils' internal networks and ensuring that devices where
using the latest version of Windows.}

\section{Completed Projects}

\dateentry{Q1 2022}{DataTable/CPS1011 Assignment}{Assignment for
Programming Principles (in C)}{I developed a very rudimentary
CLI spreadsheet program. During development I followed good C
programming practices.}

\section{Pending Projects}

\dateentry{Q3 2021 -- NOW}{Front-end Web Development}{Voluntary
work for the Malta Mathematical Society (MMS)}{I am developing a
simple website for the MMS.}

\dateentry{2020 -- NOW}{Java and Python Development}{Development
of a Propositional Logic Calculator}{Firstly, I wrote the
calculator in Python and then, I rewrote it in Java. My latest
efforts include generating random valid
implications/equivalences.}

\section{Education}

\dateentry{Q4 2021 -- NOW}{B.Sc. (Hons) in Mathematics and
Computer Science}{University of Malta}{I am in my first year of
the above-mentioned course.}

\dateentry{2019 -- 2021}{MATSEC Matriculation Certificate}{St.
Aloysius' College, Sixth From}{Advanced: Pure Mathematics (A),
Computing (A). Intermediate: Physics (A), Philosophy (A),
English (B), Systems of Knowledge (A).}

\dateentry{2014 -- 2019}{MATSEC Secondary Education Certificate
(SEC)}{St. Benedict's College, Kirkop Secondary School}{12
MATSEC O-Levels.}

\section{Skills}

\begin{center}
\setlength{\tabcolsep}{20pt}
  \begin{tabular}{c||c}
    \ulbf{Programming Languages} & \ulbf{Markup Languages}\\
    C, Java, Python & HTML and CSS, Markdown,
    \textrm{\LaTeX}\\\\
    \ulbf{Workflow} & \ulbf{Operating Systems}\\
    Tiling Window Managers, Neovim, Terminals & Linux, macOS\\\\
    \ulbf{Other Languages} & \\
    Bash, JavaScript, Rust, \CC & 
  \end{tabular}
\end{center}

\section{Areas of Interest}

\ulsubsection{Low-level/Systems Programming}

I am interested in this type of programming because of the nuance and
subtleties of the platform which the programmer has to keep in mind. This helps
build a greater understanding of hardware and operating systems.

\ulsubsection{Reverse Engineering}

I find this area to be similar to low-level programming. It forces you to learn
about a large number of things, such as: how compilers produces binaries,
assembly, etc.

\ulsubsection{Compilers and Languages}

This is an intriguing area because I am fascinated with code optimization and
the creation of programming languages. Currently, I am following the
development of Zig. A programming language meant to be a better C.

\section{Links}

\textbf{Malta Mathematical Society Website Repository}

$\bullet$ \url{https://github.com/maltamathsoc/maltamathsoc.github.io}

\textbf{Python Logic Calculator Repository}

$\bullet$ \url{https://github.com/girogio/LogicCalculator}

\textbf{Java Logic Calculator Repository}

$\bullet$ \url{https://github.com/JuanScerriE/LogicCalculator}

\textbf{DataTable/CPS1011 Assignment Repository}

$\bullet$ \url{https://gitlab.com/JuanScerriE/cps1011-assignment}

\begin{center}
\vspace{1em}\rule{.25\textwidth}{0.5pt}
\end{center}

\end{document}
